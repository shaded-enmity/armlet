\documentclass{article}%
\usepackage{tikz}%
%
\usetikzlibrary{chains,backgrounds,calc}%
\begin{document}%
%
\xdef\objectname{}%
\xdef\objectcolor{}%
\gdef\objectprops#1#2{%
  \xdef\objectname{#1}%
  \xdef\objectcolor{#2}%
}%
%
\gdef\addobject#1#2{%
  \pgfkeys{/inst/#1/ref name/.initial={aB}, /inst/#1/ref color/.initial={bA}}%
  \pgfkeys{/inst/#1/ref name={#1}, /inst/#1/ref color={#2}}%
}%
%
\pgfkeys{/tikz/fetch name/.code={\pgfkeys{/inst/#1/ref name}}}%
\pgfkeys{/tikz/fetch color/.code={\pgfkeys{/inst/#1/ref color}}}%
\pgfkeys{/tikz/bit header/.default=0, /tikz/bit header/.initial=0}%
\pgfkeys{/tikz/description/.default=0, /tikz/description/.initial=0}%
\pgfkeys{/tikz/counters/.code args={#1 and #2}{\addtocounter{#1}{1} \addtocounter{#2}{1}}}%
\pgfkeys{/tikz/bitnode/.code args={#1 and #2}{}}%
\pgfkeys{/tikz/bitnode/.add style={on chain=vert,below=of #1}{}}%
\pgfkeys{/tikz/bitnode/.append code={\ifnum #2=1 \else \draw[thick] (\tikzchaincurrent.north east) -- +(0,-0.1); \draw[thick] (\tikzchaincurrent.south east) -- +(0,0.1); \fi}}%
\pgfkeys{/tikz/add ticks/.style={execute at end scope={\addticks}}}%
\pgfkeys{/tikz/shade mix/.style={fill=black!#1}}%
\pgfkeys{/tikz/color mix/.style={fill=blue!#1!green!50}}%
\pgfkeys{/tikz/reset counter/.code={\setcounter{#1}{0}}}%
%
% INIT MACROS
%
\def\bitlabel#1#2#3#4{\ifnum #2=2\ifnum #1=1 #1\else\ifnum #1=#3 #1\fi\fi\else#4\fi}%
\def\addticks{\ifnum \theparts=1\else\draw[thick] (vert-end.north east) -- +(0,-0.1); \draw[thick] (vert-end.south east) -- +(0,0.1);\fi}%
%
%
% CREATE COUNTERS
\newcounter{iterator}%
\newcounter{parts}%
%
\newenvironment{instruction}[1]{%
	%
	% INIT COUNTERS
	\setcounter{iterator}{0}%
	\setcounter{parts}{0}%
	%
	%
	% KEYS / STYLES
	\pgfkeys{/tikz/bit header={#1}}%
	\tikzstyle{partscope} = [start branch=bp going left, node distance=-0.0275cm,%
			minimum width=0.55cm, minimum height=1.25cm,%
	      		start chain=vert,every on chain/.style={}]%
	\tikzstyle{headernode} = [on chain=master, minimum height=0.5cm]%
	%
	%
	\pgfkeys{%
	  /object/.cd, bit coverage/.initial=2, show overlay/.initial=0, node text/.initial={}, var type/.initial=0,%
	  name/.initial={}, bits/.initial=0, name overlay/.initial={}, all bits/.style={bit coverage=1}, edge bits/.style={bit coverage=2},%,
	  bitpattern/.style={var type=0}, register/.style={var type=1}, opcode/.style={var type=2}, named/.style={var type=4}%
	}%
	%
	\pgfkeys{%
	  /mnemonics/.cd, mnemonics name/.code={\node[on chain=mnem, rounded corners=1mm, fill=black!30] at ($(var-title.south) + (-0.85,-0.35)$) {\texttt{##1}};},%
	  operand/.code={\node[on chain=mnem, rounded corners=1mm,fill=cyan!45]{\texttt{##1}};},%
	  variant/.code={\node[on chain=mnem, rounded corners=1mm,fill=blue!30]{\texttt{##1}};},%
	  optional/.code={\node[on chain=mnem, rounded corners=1mm,fill=green!25]{\texttt{##1}};},%
	  inner variant/.code={\node[on chain=mnem, rounded corners=1mm,fill=blue!25]{\texttt{##1}};},%
	  open bracket/.code={\node[on chain=mnem,rounded corners=1mm, fill=black!15]{\texttt{[}};},%
	  close bracket/.code={\node[on chain=mnem, rounded corners=1mm,fill=black!15]{\texttt{]}};},%
	  open curly/.code={\node[on chain=mnem,rounded corners=1mm, fill=black!10]{\texttt{\{}};},%
	  close curly/.code={\node[on chain=mnem,rounded corners=1mm, fill=black!10]{\texttt{\}}};},%
	  comma/.code={\node[on chain=mnem,rounded corners=1mm, text depth=-0.15cm, fill=black!10]{\texttt{,}};},%
	  pause/.code={\node[on chain=mnem,rounded corners=1mm, fill=black!10]{\texttt{}};}
	}%
	%
	%
	\pgfkeys{%
		/code/.cd,
	}%
	%
	%
	\newcommand*{\newmnemonics}[1][]{\mnem[##1]};%
	\def\mnem[##1]##2;{%
		\pgfkeys{/tikz/variants/.append code={%
			\begin{scope}[start chain=mnem going right, every node/.style={minimum height=0.6cm,node distance=0.3cm and -0.02555cm}]%
				\node[on chain=variants, minimum size=0cm, name=mnemstart, yshift=+5mm]{};%
				\path[fill=white!90!black] ($(variants-end.north west)-(1.34cm,0.2)$) rectangle ($(variants-end.north west)-(header-anchor-start)-(1.02cm,2.8)$);%
				\path[thick,color=black!50] ($(variants-end.west)-(1.34cm,0.8)$) edge ($(variants-end.south west)-(header-anchor-start)-(+1.02,2.215)$);%
				\pgfkeys{/mnemonics/.cd, mnemonics name=##2, pause, ##1}%
				\node[on chain=variants, minimum size=0cm, yshift=-0.2cm, name=var-title]{};%
			\end{scope}%
			%\node[circle, fill=red!60, minimum size=2pt] at (mnem-begin) {};
			%\node[circle, fill=green!60, minimum size=2pt] at (var-title) {};
		}}%
	};
	%
	%
	% New Variant
	% \newvariant <title> <condition-node> <condition-equals>
	%
	\newcommand{\newvariant}[3]{%
		\pgfkeys{/tikz/variants/.append code={%
			\path[fill=black!20!white] ($(variants-end.north west)-(1.34cm,0.175)$) rectangle ($(variants-end.north west)-(header-anchor-start)-(1.02cm,2.6)$);
			\path[thick,color=black!50] ($(variants-end.west)-(1.34cm,0.6)$) edge ($(variants-end.south west)-(header-anchor-start)-(+1.02,2.03)$);
			\node[anchor=west,name=var-title, on chain=variants]{\texttt{ ##1} $($};%
			\node[name=var-cond,right=of var-title, rounded corners=1mm, fill=\pgfkeysvalueof{/inst/##2/ref color}]{\texttt{\pgfkeys{/tikz/fetch name={##2}}}};%
			\node[right=of var-cond, name=variant-anchor]{\texttt{##3})};%
		}}%
	};%
	%
	%
	\newcommand*{\addpart}[1][]{\@object[##1]};%
	\def\@object[##1]##2;{%
	 \begingroup%
	  %
	  \def\vtempty{}%
	  \pgfkeys{/object/.cd, ##1, node text=##2}%
	  %
	  \pgfkeysgetvalue{/object/node text}{\nodetext}%
	  \ifx\nodetext\pgfutil@empty\def\nodetext{0}\fi%
	  %
	  \pgfkeysgetvalue{/object/name}{\objname}%
	  \ifx\objname\vtempty\edef\objname{null}\fi%
	  %
	  \pgfkeysgetvalue{/object/bits}{\nbits}%
	  \ifx\nbits\pgfutil@empty\def\nbits{0}\fi%
	  %
	  \pgfkeysgetvalue{/object/bit coverage}{\coverage}%
	  \ifx\coverage\pgfutil@empty\def\coverage{0}\fi%
	  %
	  \pgfkeysgetvalue{/object/name overlay}{\objcolor}%
	  \ifx\objcolor\pgfutil@empty\edef\objcolor{0}\fi%
	  %
	  \pgfkeysgetvalue{/object/var type}{\vartype}%
	  \ifx\vartype\pgfutil@empty\def\vartype{0}\fi%
	  %
	  \xdef\objmacro{\noexpand\addobject {\objname}{\objcolor}}\aftergroup\objmacro	   
	  %
	  \def\vtbitpattern{0}%
	  \ifx\vartype\vtbitpattern%
	  	\edef\bitrange{\nodetext}%
		\edef\hues{15 : 20}%
	  	\def\drawbox{\draw[thick] (vert-end.north west) rectangle (vert-begin.south east);}%
	  	\def\nodeeval{\texttt{\bit}}
		\def\mixkey{shade mix}%
	  \else%
	  	\edef\bitrange{1,...,\nbits}%
		\edef\hues{25 : 30}%
	  	\def\drawbox{\draw[thick] (vert-end.north west) rectangle (vert-begin.south east) node[midway] {\nodetext};}%
	  	\def\nodeeval{}%
		\def\mixkey{color mix}%
	  \fi%
	  %%  
	  \begin{scope} [partscope, reset counter=parts]%%
	  	\foreach\bit[evaluate=\rc using {!Mod(\theiterator,2)==0?\hues}] in \bitrange%%
	  	{%%
		  	\begin{scope}[counters=iterator and parts, add ticks]%%
	  			\node[headernode] {\footnotesize\texttt{\bitlabel{\theparts}{\coverage}{\nbits}{\theiterator}}};%
	  			\node[bitnode={\tikzchaincurrent} and {\nbits}, {\mixkey}=\rc] {\nodeeval};%
	  		\end{scope}%%
	  	};%%
	    \ifx\objcolor\vtempty\else%%
	  		\draw[thick,fill=\objcolor] ($(vert-end.west) + (0,-0.185)$) rectangle ($(vert-begin.south east) + (0,0.005)$) node[midway] (\objname) {\footnotesize\texttt{\objname}};%%
	  	\fi%%
	  	\drawbox%
	  	%
	  \end{scope}%		  
	\endgroup };%
	%%	
	\begin{scope}[start chain=master going left]%
}{%
		\node[anchor=west] at ($(master-end.south east) + (-0.65,0.75)$) {\texttt{\pgfkeysvalueof{/tikz/bit header}}};%
		\draw[thick] (master-end.north west) rectangle ($(master-begin.south east) + (0,0.03)$);%
		
		\coordinate (header-anchor-start) at ($(vert-end.south west) + (-0.005,-0.00)$);%
		\coordinate (header-anchor-end) at ($(header-anchor-start) - (master-end.west) + (0.3,0)$);%
		
		%\node[circle, fill=red!60, minimum size=2pt] at (header-anchor-end) {};
		%\node[circle, fill=blue!60, minimum size=2pt] at (header-anchor-start) {};
		
		\begin{scope}[node distance=-0.1cm and -0.0675cm, start chain=variants going below]%
			\node[anchor=west,name=anchor-root,on chain=variants] at ($(vert-end.south east) + (0.75,0.05)$) {};%
			\pgfkeys{/tikz/variants}%
		\end{scope}%	
		\draw[thick,color=black!100] (header-anchor-start) rectangle ($(mnem-begin.south west) - (master-end.south west) + (0.095, -0.34)$);
	\end{scope}%
}%
%
%
\scalebox{0.9}{
\begin{tikzpicture}%
	\begin{instruction}{85. LDR (register)}%
	%
		\addpart[bits=5,register] {Rd};%	
		\addpart[bits=5,register] {Rn};%
		\addpart[bits=2] {0, 1};%
		\addpart[bits=1,opcode] {S};%
		\addpart[name=opc1,opcode,bits=3,name overlay=red!30] {option};%
		\addpart[bits=5,register] {Rm};%
		\addpart[bits=9] {1, 1, 0, 0, 0, 0, 1, 1, 1};%
		\addpart[name=size,bits=2,name overlay=orange!50] {x,1};%	
		%
		\newvariant{32-bit variant}{size}{== 10};%
		%
		\newmnemonics[operand={<Wt>}, comma, open bracket, optional={<Xn|SP>}, comma, optional={<R>}, optional={<m>},
		   open curly, variant={<extend>}, open curly, comma, inner variant={<amount>}, 
		   close curly, close curly, close bracket]{LDR};
		%
		\newvariant{64-bit variant}{size}{== 11};%
		%
		\newmnemonics[operand={<Xt>}, comma, open bracket, optional={<Xn|SP>}, comma, optional={<R>}, optional={<m>},
		   open curly, variant={<extend>}, open curly, comma, inner variant={<amount>}, 
		   close curly, close curly, close bracket]{LDR};
		%
		%\addcode[boolean, var={wback}, assign, false token, end line,
		%		  boolean, var={postindex}, assign, false token, end line,
		%         integer, var={scale}, assign, uint cast={size}, end line,
		%		  if token, bit var=1 of option, equals, quoted={0}, then token, bad encoding, semicolon token, comment={sub-word index}, finish line,
		%		  typed={ExtendType}, var={extend_type}, assign, call={DecodeRegExtend}, one arg={option}, end line,
		%		  integer, var={shift}, assign, if token, var={S}, equals, quoted={1}, then token, var={scale}, else token, immediate={0}, end line]
		%boolean wback = FALSE; 
		%boolean postindex = FALSE; 
		%integer scale = UInt(size); 
		%if option<1> == '0' then UnallocatedEncoding(); // sub-word index 
		%ExtendType extend_type = DecodeRegExtend(option); 
		%integer shift = if S == '1' then scale else 0; 
	%
	%
	\end{instruction}%
%
\end{tikzpicture}%
}
\end{document}%